\documentclass{ctexart}
% \special{dvipdfmx:config z 0}
% \setCJKmainfont{SimHei}[BoldFont=SimHei, ItalicFont=KaiTi]
% pdf 转 svg 的话就将上面一行取消注释.
\pagestyle{empty}
\usepackage[english]{babel}
\usepackage{amsfonts, amsmath, amssymb, amscd, amsthm}
\usepackage{mathrsfs, bm, bbm}
\usepackage{xparse}
\usepackage{physics}
\usepackage{esint}
\usepackage{xypic}
\usepackage{tikz}
\usepackage{graphicx, subfigure}
\usepackage{enumerate}
\usepackage[textwidth=444bp,vmargin=2.5cm]{geometry}%设置页边距
\usepackage{array} %主要是增加列样式选项



% 设置
\RequirePackage{xparse, amsmath, physics}
\renewcommand{\baselinestretch}{1.5}
\renewcommand{\indent}{\hspace{2em}}
\newcommand{\aneg}[1]{\hspace{-0.75em}&#1&\hspace{-0.75em}}
\newcommand{\aneq}{\aneg{=}}
% 上述指令用于在使用 array 环境时调整等号左右间距
\newcommand{\eqs}{\quad\;}
% 上述指令用于 align 环境中, &= 的换行对齐时调整第二行位置
\renewcommand{\d}{\displaystyle}



% 字符
\renewcommand{\i}{\mathrm{i}}
\renewcommand{\j}{\mathrm{j}}
\newcommand{\e}{\textup{e}}
\newcommand{\ve}{\varepsilon}
\newcommand{\Beta}{\mathit{B}}
\newcommand{\omicron}{\mathit{o}}
\newcommand{\Omicron}{\mathit{O}}	% 这三个无所谓啦

\newcommand{\D}{\mathbb{D}}
\newcommand{\E}{\mathbb{E}}
\newcommand{\F}{\mathbb{F}}
\newcommand{\K}{\mathbb{K}}
\renewcommand{\L}{\mathbb{L}}	% 已有定义, 不过我不知道原定义的作用. 如有需求则取消注释.
\newcommand{\J}{\mathbb{J}}



% 上下标
\newcommand{\trans}{^\mathrm{T}}
\newcommand{\inv}{^{-1}}
\newcommand{\adj}[1]{^{\pqty{#1^*}}}
%\newcommand{\vector}[1]{\overrightarrow{#1}}	% 使用 typora 时取消注释



% 特定内容
\newcommand{\ccdots}{\cdot\cdots\cdot}
\newcommand{\oneton}{1,2,\cdots,n}
\newcommand{\oneto}[1]{1,2,\cdots,#1}

\newcommand{\ssto}[3]{#1_1 #3 #1_2 #3 \cdots #3 #1_{#2}}
\newcommand{\ssup}[3]{#1^1 #3 #1^2 #3 \cdots #3 #1^{#2}}
\newcommand{\soneto}[2]{\ssto{#1}{#2}{,}}
\newcommand{\splus}[2]{\ssto{#1}{#2}{+}}

\newcommand{\aqty}[1]{\expval{#1}}
\newcommand{\pbqty}[1]{\left(#1\right]}
\newcommand{\bpqty}[1]{\left[#1\right)}
% 也可以用下面三行代码, 只不过 typora 中无法使用.
%\DeclareDocumentCommand\aqty{ l m }{\braces#1{\langle}{\rangle}{#2}}
%\DeclareDocumentCommand\pbqty{ l m }{\braces#1{\lparen}{\rbrack}{#2}}
%\DeclareDocumentCommand\pbqty{ l m }{\braces#1{\lbrack}{\rparen}{#2}}
\newcommand{\floor}[1]{\left\lfloor#1\right\rfloor}
\newcommand{\ceil}[1]{\left\lceil#1\right\rceil}

% matrix, pmatrix 和 cases 已经有了
% array, align, equation 等没必要写, 它们的内容一般比较多, 用 \begin 和 \end 更显眼一点也不错. 而且我在 typora 中的习惯是, 第一个 \begin 后不缩进, 因为
% 1. 一个 LaTeX 环境一般只有一个 \begin{align}, 否则会有报错 "Erroneous nesting of equation structures";
% 2. 一般一个 LaTeX 环境只放一个或一组公式, 否则点进去查看时, 代码和渲染不能同屏显示;
% 3. 软件页面一般不到半屏, 第一次缩进很占空间.
%\newcommand{\bmatrix}[1]{\begin{bmatrix}#1\end{bmatrix}}
%\newcommand{\Bmatrix}[1]{\begin{Bmatrix}#1\end{Bmatrix}}
%\newcommand{\vmatrix}[1]{\begin{vmatrix}#1\end{vmatrix}}
%\newcommand{\Vmatrix}[1]{\begin{Vmatrix}#1\end{Vmatrix}}



% 下面几个只是为了方便一点点而已
\newcommand{\dx}{\dd{x}}
\newcommand{\dy}{\dd{y}}
\newcommand{\dz}{\dd{z}}
\newcommand{\dt}{\dd{t}}
\newcommand{\ds}{\dd{s}}
\newcommand{\dr}{\dd{r}}

% 如果只使用 \dd{x}\dd{y} 的话, 中间会有多余的间隔.
\newcommand{\df}{\dd}
\newcommand{\ddf}[2]{\,\mathrm{d}#1\mathrm{d}#2}	% 微分形式 differential form
\newcommand{\dddf}[3]{\,\mathrm{d}#1\mathrm{d}#2\mathrm{d}#3}

\newcommand{\dxdy}{\ddf{x}{y}}
\newcommand{\dydz}{\ddf{y}{z}}
\newcommand{\dzdx}{\ddf{z}{x}}
\newcommand{\dudv}{\ddf{u}{v}}
\newcommand{\drdt}{\ddf{r}{\theta}}
\newcommand{\dxdydz}{\dddf{x}{y}{z}}

% 矩阵的宏指令
\newcommand{\pmcmn}[3]{\begin{pmatrix}
	#1_{11} & #1_{12} & \cdots & #1_{1#3} \\
	#1_{21} & #1_{22} & \cdots & #1_{n#3} \\
	\vdots & \vdots && \vdots \\
	#1_{#2 1} & #1_{#2 2} & \cdots & #2_{n#3} \\
\end{pmatrix}}

\newcommand{\pmc}[1]{\pmcmn{#1}{n}{n}}

\newcommand{\pvcn}[2]{\begin{pmatrix}
	#1_1 \\ #1_2 \\ \vdots \\ #1_{#2}
\end{pmatrix}}

\newcommand{\pvc}[1]{\pvcn{#1}{n}}

\newcommand{\pto}{\overset{P}{\to}}



% 函数名
\renewcommand{\char}{\operatorname{char}}	% 由于已存在此命令, 不可使用 DeclareMathOperator
\renewcommand{\r}{\operatorname{r}}
\DeclareMathOperator{\st}{s.t.\,}	% 虽然不是函数名, 但用了这个指令就放这儿了.
\DeclareMathOperator{\diag}{diag}	% 不需要定义太多, 一个文件里用到什么定义什么,
\DeclareMathOperator{\Ker}{Ker}		% 毕竟特殊的函数名太多太多了.
\DeclareMathOperator{\Aut}{Aut}		% 便捷与效率的权衡.
\DeclareMathOperator{\Inn}{Inn}
\DeclareMathOperator{\GL}{GL}
\DeclareMathOperator{\SL}{SL}
\DeclareMathOperator{\stab}{stab}
\DeclareMathOperator{\orb}{orb}
\DeclareMathOperator{\lcm}{lcm}
\DeclareMathOperator{\Var}{Var}
\DeclareMathOperator{\Cov}{Cov}
\DeclareMathOperator{\Corr}{Corr}
\DeclareMathOperator{\rot}{rot}
\DeclareMathOperator{\sgn}{sgn}
%\DeclareMathOperator{\span}{span}

\def\arsh{\ensuremath{\operatorname{arsh}}}
\def\arch{\ensuremath{\operatorname{arch}}}
\def\arth{\ensuremath{\operatorname{arth}}}



% 运算符
\newcommand{\capop}{\displaystyle\mathop\cap\limits}
\newcommand{\cupop}{\displaystyle\mathop\cup\limits}
\newcommand{\oplusop}{\mathop\oplus\limits}
\newcommand{\otimesop}{\mathop\otimes\limits}
\newcommand{\bigoplusop}{\mathop\bigoplus\limits}
\newcommand{\bigotimesop}{\mathop\bigotimes\limits}
% emmmm, 才发现, 上面几个宏没有必要写,
% 直接用 \bigcap, \bigcup, \bigoplus, \bigotimes 就可以了

\newcommand{\dint}{\displaystyle\int}
\newcommand{\inti}{\dint_{-\infty}^{+\infty}}
\newcommand{\intoi}{\dint_0^{+\infty}}

\newcommand{\intl}{\displaystyle\int\limits}
\newcommand{\iintl}{\displaystyle\iint\limits}
\newcommand{\iiintl}{\displaystyle\iiint\limits}

\newcommand{\dsum}{\displaystyle\sum}
\newcommand{\csum}[1]{\dsum_{#1=1}^\infty}
\newcommand{\nsum}{\csum{n}}
\newcommand{\ksum}{\csum{k}}
\newcommand{\nosum}{\dsum_{n=0}^\infty}
\newcommand{\insum}{\dsum_{i=1}^n}
\newcommand{\knsum}{\dsum_{k=1}^n}

\newcommand{\dprod}{\displaystyle\prod}
\newcommand{\nprod}{\dprod_{n=1}^\infty}
\newcommand{\noprod}{\dprod_{n=0}^\infty}
\newcommand{\inprod}{\dprod_{i=1}^n}

\newcommand{\liml}{\lim\limits}
\newcommand{\xlim}{\lim\limits_{x\to x_0}}
\newcommand{\nlim}{\lim\limits_{n\to\infty}}
\newcommand{\clim}[1]{\lim\limits_{#1\to\infty}}
\newcommand{\ulim}{\overline\lim\limits_{n\to\infty}}
\newcommand{\dlim}{\underline\lim\limits_{n\to\infty}}
% 注意这里的 d 是 down, 而不是 displaystyle



% 缩写
%\newcommand{\bm}[1]{\boldsymbol{#1}}		% 使用 typora 时取消注释
\newcommand{\LRA}{\Leftrightarrow}
\newcommand{\RLA}{\Leftrightarrow}
\newcommand{\LA}{\Leftarrow}
\newcommand{\RA}{\Rightarrow}

\newcommand{\lra}{\leftrightarrow}
\newcommand{\rla}{\leftrightarrow}
\newcommand{\la}{\leftarrow}
\newcommand{\ra}{\rightarrow}

\newcommand{\QRLA}{\quad\RLA\quad}
\newcommand{\QRA}{\quad\RA\quad}
\newcommand{\LLRA}{\Longleftrightarrow}

\newcommand{\QNRA}{\quad\nRightarrow\quad}
\newcommand{\qnra}{\quad\nrightarrow\quad}



% 图形符号
%\newcommand{\qed}{\quad\square}
\renewcommand{\parallel}{\mathrel{/\mskip-2.5mu/}}
\newcommand{\paralleleq}{\;
	\begin{tikzpicture}[line width=0.1ex, line cap=round]
		\draw (0, 0) -- (1.4ex, 0) [yshift=0.3ex] (0, 0) -- (1.4ex, 0);
		\draw (0.4ex, 0.5ex) -- ++(70:1.2ex) [xshift=0.35ex] (0.4ex, 0.5ex) -- ++(70:1.2ex);
	\end{tikzpicture}
\;}
\newcommand{\rt}{\matrm{Rt}\triangle}



% Xparse 宏包带参宏指令
% 注: 以下四个指令可用 physics 中的 \dv 与 \pdv 代替
\NewDocumentCommand{\ider}{o m}{
	\IfNoValueT{#1}{\frac{\d}{\d#2}}
	\IfNoValueF{#1}{\frac{\d^#1}{\d#2^#1}}
}
\NewDocumentCommand{\der}{o m m}{
	\IfNoValueT{#1}{\frac{\d#2}{\d#3}}
	\IfNoValueF{#1}{\frac{\d^#1#2}{\d#3^#1}}
}
\NewDocumentCommand{\ipd}{o m}{
	\IfNoValueT{#1}{\frac{\partial}{\partial#2}}
	\IfNoValueF{#1}{\frac{\partial^#1}{\partial#2^#1}}
}
\NewDocumentCommand{\pd}{o m m}{
	\IfNoValueT{#1}{\frac{\partial#2}{\partial#3}}
	\IfNoValueF{#1}{\frac{\partial^#1#2}{\partial#3^#1}}
}

% 注: 以下两个指令在 typora 中无法使用,因为无法使用 xparse 宏包
\DeclareDocumentCommand{\DD}{ o g d() }{ % difference operator \Delta
	% o: optional n for nth difference
	% g: optional argument for readability and to control spacing
	% d: long-form as in d(cos x)
	\IfNoValueTF{#2}{
		\IfNoValueTF{#3}
		{\Delta\IfNoValueTF{#1}{}{^{#1}}}
		{\mathinner{\Delta\IfNoValueTF{#1}{}{^{#1}}\argopen(#3\argclose)}}
	}
	{\mathinner{\Delta\IfNoValueTF{#1}{}{^{#1}}#2} \IfNoValueTF{#3}{}{(#3)}}
}

\DeclareDocumentCommand{\DV}{ s o m g d() }
{ % Total difference
	% s: star for \flatfrac flat difference
	% o: optional n for nth difference
	% m: mandatory (x in df/dx)
	% g: optional (f in df/dx)
	% d: long-form d/dx(...)
	\IfBooleanTF{#1}
	{\let\fractype\flatfrac}
	{\let\fractype\frac}
	\IfNoValueTF{#4}
	{
		\IfNoValueTF{#5}
		{\fractype{\Delta \IfNoValueTF{#2}{}{^{#2}}}{\Delta #3\IfNoValueTF{#2}{}{^{#2}}}}
		{\fractype{\Delta \IfNoValueTF{#2}{}{^{#2}}}{\Delta #3\IfNoValueTF{#2}{}{^{#2}}} \argopen(#5\argclose)}
	}
	{\fractype{\Delta \IfNoValueTF{#2}{}{^{#2}} #3}{\Delta #4\IfNoValueTF{#2}{}{^{#2}}}}
}
\DeclareDocumentCommand\DDV{}{\displaystyle\DV}

% 如果是 typora,就注释了上面这几个,凑合着用下面这三个
%\newcommand{\DD}{\Delta}
%\newcommand{\DV}[2]{\dfrac{\DD#1}{\DD#2}}
%\newcommand{\nDV}[3]{\dfrac{\DD^{#1}#2}{\DD#3^{#1}}}


% displaytsyle 的指令缩写
% 下面两个为了 LaTeX 和 typora 的通用性, 就没用 \DeclareDocumentCommand 了.
\newcommand{\ddv}{\displaystyle\dv}
\newcommand{\dpdv}{\displaystyle\pdv}
%\newcommand{\dbinom}{\displaystyle\binom}
