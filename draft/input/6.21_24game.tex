\[
\begin{aligned}
	f(a, b, c, d)
	&= \pqty{ \bqty{\Bqty{\vqty{a}}}! + \bqty{\Bqty{\vqty{b}}}! + \bqty{\Bqty{\vqty{c}}}! + \bqty{\Bqty{\vqty{d}}}! }!
	\\
	f(x_1, x_2, \cdots, x_n)
	&= \left\lceil
		\sec(\sqrt{ \tan(
			\left\lfloor
				\Bqty{\vqty{x_1}}
			\right\rfloor !
		) }) \right\rceil ! +
	\sum_{i=2}^n {
		\left\lfloor
			\Bqty{\vqty{x_i}}
		\right\rfloor
	}
	\\
	f(x_1, x_2, \cdots, x_n)
	&= \left\lceil \sinh\pqty{
		\left\lceil \tan\pqty{
			\left\lfloor
				\Bqty{\vqty{x_1}}
			\right\rfloor !
		} \right\rceil
	} \right\rceil ! +
	\sum_{i=2}^n {
		\left\lfloor
			\Bqty{\vqty{x_i}}
		\right\rfloor
	}
\end{aligned}
\]

某不讲理的算24点万能公式, 复数范围都可以算. 后两个公式适用于任意个数元素,使用时将求和展开即可.
其中 $ \vqty{} $ 是取模运算, $ \Bqty{} $ 是取小运算, $ \lfloor\rfloor $ 是取整运算, $ \lceil\rceil $ 是取顶运算.
