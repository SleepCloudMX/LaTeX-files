%$\Sigma$ 是球面 $x^2+y^2+z^2=1$ 的外侧,则 $
\displaystyle \oiint\limits_\Sigma
\frac{ \dd{y}\dd{z} }{x} +
\frac{ \dd{z}\dd{x} }{y} +
\frac{ \dd{x}\dd{y} }{z} =
$ ?
%一道曲面积分,使用一种方法可以解出来,但另一种方法却面临无穷积分

%\begin{align*}
	u =u(x) &, v = v(x)
	\\
	F(x,u,v) &\equiv \int_u^v f(x,t) \dd{t}
	\\
	\dv{x} F(x,u,v) &= \pdv{F}{x} + \pdv{F}{u} \dv{u}{x} + \pdv{F}{v} \dv{v}{x}
	\\
	&= \int_u^v f_x'(x,t) \dd{t} + v'f(x,u) - u'f(x,v)
\end{align*}
%二元函数偏积分的导数

%\begin{align*}
	\qq{none:}	&aA\alpha\Lambda \\
	\qq{mathbf:} &\mathbf{aA\alpha\Lambda} \\
	\qq{boldsymbol:} &\boldsymbol{aA\alpha\Lambda} \\
	\qq{vb:} &\vb{aA\alpha\Lambda} \\
	\qq{vb*:} &\vb*{aA\alpha\Lambda} \\
	\qq{bm:} &\bm{aA\alpha\Lambda} \\
	\qq{bold:} &\bold{aA\alpha\Lambda} \\
\end{align*}
%测试各种粗体字

%\begin{table}[h]
	\centering
	\newcommand{\tabincell}[2]{\begin{tabular}{@{}#1@{}}#2\end{tabular}}
	\begin{tabular}{|c|c|c|}
		\hline
		\tabincell{c}{\\ $\displaystyle\int_0^x 2t\dd{t} = \int_0^x \dd{t^2} = x^2$ \\ \textbf{守序善良} } &
		\tabincell{c}{\\ $\displaystyle \int_0^d 2x\dd{x} = \int_0^d \dd{x^2} = d^2$ \\ \textbf{中立善良}} &
		\tabincell{c}{\\ $\displaystyle\int_0^x 2tdt = \int_0^x dt^2 = x^2$ \\ \textbf{混乱善良} }
		\\ \hline
		\tabincell{c}{\\ $\displaystyle\int_0^x 2x\dd{x} = \int_0^x \dd{x^2} = x^2$ \\ \textbf{守序中立}} &
		\tabincell{c}{\\ $\displaystyle\int_0^d 2d\dd{d} = \int_0^d \dd{d^2} = d^2$ \\ \textbf{绝对中立}} &
		\tabincell{c}{\\ $\displaystyle\int_0^d 2ddd = \int_0^d dd^2 = d^2$ \\ \textbf{混乱中立}}
		\\ \hline
		\tabincell{c}{\\ $\displaystyle\int_0^x 2d\dd{d} = \int_0^x \dd{d^2} = x^2$ \\ \textbf{守序邪恶}} &
		\tabincell{c}{\\ $\displaystyle\int_0^d 2d\dd{d} = \int_0^d \dd{dd} = dd$ \\ \textbf{中立邪恶}} &
		\tabincell{c}{\\ $\displaystyle\int_0^d 2ddd = \int_0^d ddd = dd$ \\ \textbf{混乱邪恶}}
		\\ \hline
	\end{tabular}
\end{table}

%九宫格阵营图,整活系列

%\[
\left\{ \begin{array}{l l}
	\dfrac{y+z}{x}, & x<0 \\
	\dfrac{x}{a+b}, & x\ge0 \\
\end{array} \right.
\]

\[
\begin{cases}
	\dfrac{y+z}{x}, & x<0 \\
	\dfrac{x}{a+b}, & x\ge0 \\
\end{cases}
\]

\[
\left\{ \begin{aligned}
	\dfrac{y+z}{x}, \quad& x<0 \\
	\dfrac{x}{a+b}, \quad& x\ge0 \\
\end{aligned} \right.
\]

\[
\left\{ \begin{matrix}
	\dfrac{y+z}{x}, & x<0 \\
	\dfrac{x}{a+b}, & x\ge0 \\
\end{matrix} \right.
\]

\begin{equation*}
	\left\{
	\begin{aligned}
		&a_n=\frac{b_{n-1}+c_{n-1}}{2} \\
		&b_n=\frac{c_{n-1}+a_{n-1}}{2},\quad(n=1,~2,~3,\cdots). \\
		&c_n=\frac{a_{n-1}+b_{n-1}}{2}
	\end{aligned}
	\right.
\end{equation*}

\[
\begin{cases}
	a_n=\dfrac{b_{n-1}+c_{n-1}}{2} \\
	b_n=\dfrac{c_{n-1}+a_{n-1}}{2}, &(n=1,~2,~3,\cdots). \\
	c_n=\dfrac{a_{n-1}+b_{n-1}}{2}
\end{cases}
\]
%各种大括号对齐方式
%注意使用 \renewcommand{\baselinestretch}{1.5}
%那样会更好看

%\[
\begin{pmatrix}
	\underbrace{\begin{matrix}
			0&&& \\ 1&0&& \\ &\ddots&\ddots& \\ &&1&0 \\
	\end{matrix}}_{k_1} &&&
	\\
	& \underbrace{\begin{matrix}
			0&&& \\ 1&0&& \\ &\ddots&\ddots& \\ &&1&0 \\
	\end{matrix}}_{k_2} &&
	\\
	&&\ddots&
	\\
	&&& \underbrace{\begin{matrix}
			0&&& \\ 1&0&& \\ &\ddots&\ddots& \\ &&1&0 \\
	\end{matrix}}_{k_s}
\end{pmatrix}
\]
%准对角阵,每个分块矩阵下用大括号表明元素个数

%\begin{align}
	&\mathrm{a} a \alpha \\
	&\mathrm{v} v \nu \\
	&\mathrm{u} u \upsilon \\
	&\mathrm{r} r \gamma \varUpsilon  \\
	&\mathrm{x} x \chi \\
	&\mathrm{k} k \kappa \\
	&\mathrm{o} o \omicron \\
	&\mathrm{w} w \omega \\
	&\phi \varphi \psi \\
	&\Phi \Psi \\
	&\zeta \xi \varepsilon \epsilon \in \\
	&\vartheta \theta \Theta \\
	&\varsigma \sigma \delta \\
	&\pi \Pi \prod \\
	&\Lambda \wedge \cap \\
	&\triangle \Delta \\
	&\nabla \grad \\
	&\emptyset \varnothing \not O \\
\end{align}
%各种相似的字母、字符

%\begin{align}
	&\mathrm{a} a &\mathbb{a} &\mathbf{a} &\mathbin{a}
	&\mathcal{a} &\mathclose{a} &\mathfrak{a} &\mathinner{a} &\mathit{a}
	&\mathopen{a} &\mathord{a} &\mathpunct{a} &\mathrel{a} &\mathring{a}
	&\mathscr{a} &\mathsf{a} &\mathstrut{a} &\mathtt{a}
	\\
	&\mathrm{A} a &\mathbb{A} &\mathbf{A} &\mathbin{A}
	&\mathcal{A} &\mathclose{A} &\mathfrak{A} &\mathinner{A} &\mathit{A}
	&\mathopen{A} &\mathord{A} &\mathpunct{A} &\mathrel{A} &\mathring{A}
	&\mathscr{A} &\mathsf{A} &\mathstrut{A} &\mathtt{A} 
\end{align}
%测试各种数学字体

%\[
\begin{aligned}
	f(a, b, c, d)
	&= \pqty{ \bqty{\Bqty{\vqty{a}}}! + \bqty{\Bqty{\vqty{b}}}! + \bqty{\Bqty{\vqty{c}}}! + \bqty{\Bqty{\vqty{d}}}! }!
	\\
	f(x_1, x_2, \cdots, x_n)
	&= \left\lceil
		\sec(\sqrt{ \tan(
			\left\lfloor
				\Bqty{\vqty{x_1}}
			\right\rfloor !
		) }) \right\rceil ! +
	\sum_{i=2}^n {
		\left\lfloor
			\Bqty{\vqty{x_i}}
		\right\rfloor
	}
	\\
	f(x_1, x_2, \cdots, x_n)
	&= \left\lceil \sinh\pqty{
		\left\lceil \tan\pqty{
			\left\lfloor
				\Bqty{\vqty{x_1}}
			\right\rfloor !
		} \right\rceil
	} \right\rceil ! +
	\sum_{i=2}^n {
		\left\lfloor
			\Bqty{\vqty{x_i}}
		\right\rfloor
	}
\end{aligned}
\]

某不讲理的算24点万能公式, 复数范围都可以算. 后两个公式适用于任意个数元素,使用时将求和展开即可.
其中 $ \vqty{} $ 是取模运算, $ \Bqty{} $ 是取小运算, $ \lfloor\rfloor $ 是取整运算, $ \lceil\rceil $ 是取顶运算.

%24点万能公式, 任意个数的数字均可, 复数也行.

%\[
\begin{aligned}
	\pqty{ \frac{a}{b} }\inv &\quad (\frac{a}{b})\inv
	\\
	\pqty{A\inv}\inv &\quad (A\inv)\inv
	\\
	\pqty{A\trans}^* &\quad (A\trans)^*
\end{aligned}
\]
%自适应大括号效果一般都更好,但是 (A\inv)\inv 就显得不太美观了(吧?);但不用自适应大括号的话,(A\trans)^* 又会很难看. 我选择自适应

%\[
\xymatrix{
	A_1 \ar[d]_{h_1} \ar[r]^{f_1}
	& B_1 \ar[d]_{h_2} \ar[r]^{g_1}
	& C_1 \ar[d]_{h_3}
	\\
	A_2 \ar[r]^{f_2}
	& B_2 \ar[r]^{g_2}
	& C_2
}
\]
%用 xy-pic 宏包画交换图

%\[
\mathop{\boxminus}\limits^{+\hspace{-0.25em}-\hspace{-0.25em}+}_{\Large-\hspace{-0.15em}+\hspace{-0.15em}-}
\]
% 用 latexlive.com 或者 typora 的话, 最后两个 0.15 改成 0.5
%手动画草

%\[
\begin{array}{ccc}
	\doteq & \equiv & \overset{\mathrm{def}}{=\hspace{-1em}=} \\
	\overset{\mathrm{d}}{=} & := & \triangleq
\end{array}
\]
%罗列各种用于定义的符号

%\begin{align*}
	\dv{x} \dv{f}{x}(x) \dv[n]{f}{x}(x) \\
	\dv{x} \dv{f(x)}{x} \dv[n]{f(x)}{x} \\
	\DV{x} \DV{f(x)}{x} \DV[n]{f(x)}{x} \\
	\dd \dd{x} \dd[n]{x} \\
	\DD \DD{x} \DD[n]{x} \\
	\ddv[n]{f(x)}{x}
\end{align*}
%physics 宏包和自己写的一些导数指令

\begin{equation}
\begin{array}{|l|l|}
	\hline
	\text{\hspace{0.167em}} & a \hspace{0.167em} b \\
	\text{\,} & a \, b \\
	\hline
	\text{\hspace{0.25em}} & a \hspace{0.25em} b \\
	\text{\;} & a \; b \\
	\text{\ } & a \ b \\
	\text{~} & a ~ b \\
	\text{\space} & a \space b \\
	\hline
	\text{\hspace{1em}} & a \hspace{1em} b \\
	\text{\quad} & a \quad b \\
	\hline
\end{array}
\end{equation}

%不同空格指令的效果