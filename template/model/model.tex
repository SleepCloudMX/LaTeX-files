\documentclass[a4paper, 11pt, UTF8]{article}
\usepackage{ctex}		%请勿将 article 与 ctex 简写成 ctexart
\usepackage{xparse}
\usepackage[english]{babel}
\usepackage{amsfonts, amsmath, amssymb, amscd, amsthm}
\usepackage{mathrsfs, bm, bbm}
\usepackage{indentfirst}
\usepackage{physics}
%\usepackage{graphicx, subfigure}

%--------------------------settings--------------------------
\usepackage{fontspec}
\setmainfont{Microsoft YaHei UI}
\renewcommand{\baselinestretch}{1.25}
\renewcommand{\qedsymbol}{\ensuremath{\square}}

%-------------------------definition-------------------------
\def\indent{\hspace{2em}}
\def\d{\textup{d}}
\def\e{\textup{e}}
\def\Beta{\textup{B}}
\def\trans{\ensuremath{^\mathrm{T}}}
\def\oneton{\{1,2,3,...,n\}}

\def\arsh{\ensuremath{\operatorname{arsh}}}
\def\arch{\ensuremath{\operatorname{arch}}}
\def\arth{\ensuremath{\operatorname{arth}}}

\NewDocumentCommand{\ider}{o m}{
	\IfNoValueT{#1}{\frac{\d}{\d#2}}
	\IfNoValueF{#1}{\frac{\d^#1}{\d#2^#1}}
}
\NewDocumentCommand{\der}{o m m}{
	\IfNoValueT{#1}{\frac{\d#2}{\d#3}}
	\IfNoValueF{#1}{\frac{\d^#1#2}{\d#3^#1}}
}
\NewDocumentCommand{\ipd}{o m}{
	\IfNoValueT{#1}{\frac{\partial}{\partial#2}}
	\IfNoValueF{#1}{\frac{\partial^#1}{\partial#2^#1}}
}
\NewDocumentCommand{\pd}{o m m}{
	\IfNoValueT{#1}{\frac{\partial#2}{\partial#3}}
	\IfNoValueF{#1}{\frac{\partial^#1#2}{\partial#3^#1}}
}

\theoremstyle{plain}
\newtheorem{thm}{定理}[section]
\newtheorem{lem}{引理}[section]
\newtheorem{prop}{命题}[section]
\newtheorem{cor}{推论}[section]

\theoremstyle{definition}
\newtheorem{defn}{定义}[section]
\newtheorem{exmp}{例子}[section]
\newtheorem*{ack}{致谢}

\theoremstyle{Remark}
\newtheorem{rem}{注释}[section]
\usepackage{listings}
\usepackage[usenames,dvipsnames]{color}

\definecolor{DarkGreen}{rgb}{0.0,0.4,0.0}

\lstloadlanguages{Matlab}
\lstset{
	language=Matlab,
	basicstyle = \fontspec{Consolas},
	basicstyle = \tt,
	frame=single,                           % single framed
	basicstyle=\small\ttfamily,
	keywordstyle=[1]\color{Blue}\bfseries,  % primitive funs in bold blue
	keywordstyle=[2]\color{Purple},         % args of funs in purple
	keywordstyle=[3]\color{Blue}\underbar,  % user funs in blue with underbar
	stringstyle=\color{Purple},             % strings in purple
	showstringspaces=false,
	identifierstyle=,
	commentstyle=\usefont{T1}{pcr}{m}{sl}\color{DarkGreen}\small,
	tabsize=4,
	% more standard MATLAB funcs
	morekeywords={sawtooth, square},
	% args of funcs
	morekeywords=[2]{on, off, interp},
	% user funcs
	morekeywords=[3]{FindESS, homework_example},
	morecomment=[l][\color{Blue}]{...},     % line continuation (...) like blue comment
	numbers=left,
	numberstyle=\tiny\color{Blue},
	firstnumber=1,
	stepnumber=1
}

\begin{document}
%----------------------------设置----------------------------
\ctexset{today=small}
\ctexset{abstractname={摘要}}
\ctexset{proofname={证明.}}
\ctexset{bibname={参考文献}}

\title{一篇论文的标题}
\author{你的名字}
\date{\today}
\maketitle
\thispagestyle{empty}

%----------------------------摘要----------------------------
\newpage
\setcounter{page}{1}
\begin{abstract}
论文的摘要。

开头段:需要充分概括论文内容,一般两到三句话即可,长度控制在三至五行。

问题一中,解决了什么问题;应用了什么方法;得到了什么结果。

问题二中,解决了什么问题;应用了什么方法;得到了什么结果。

问题三中,解决了什么问题;应用了什么方法;得到了什么结果。

结尾段:可以总结下全文,也可以介绍下你的论文的亮点,也可以对类似的问题进行适当的推广。
\end{abstract}

\begin{keyword}
	关键词 1 \quad 关键词 2
\end{keyword}

%----------------------------引言----------------------------
%\setcounter{section}{-1}
%\section{引言}
%	论文的引言。

%--------------------------问题重述--------------------------
\newpage
\section{问题重述}    
\textbf{重述问题,简明扼要,忌抄原题。}

数学建模比赛论文是要我们解决一道给定的问题,所以正文部分一般应从问题重述开始,一般确定选题后就可以开始写这一部分了。

这部分的内容是将原问题进行整理,将问题背景和题目分开陈述即可,所以基本没啥难度。

本部分的目的是要吸引读者读下去,所以文字不可冗长,内容选择不要过于分散、琐碎,措辞要精练。

注意:在写这部分的内容时,绝对不可照抄原题!(论文会查重)

应为:在仔细理解了问题的基础上,用自己的语言重新将问题描述一遍。语言需要简明扼要,没有必要像原题一样面面俱到。

\section{问题分析}
\subsection{问题一的分析}
从实际问题到模型建立是一种从具体到抽象的思维过程,问题分析这一部分就是沟通这一过程的桥梁,因为它反映了建模者对于问题的认识程度如何,也体现了解决问题的雏形,起着承上启下的作用,也很能反应出建模者的综合水平。

这部分的内容应包括:题目中包含的信息和条件,利用信息和条件对题目做整体分析,确定用什么方法建立模型,一般是每个问题单独分析一小节,分析过程要简明扼要, 不需要放结论。

建议在文字说明的同时用图形或图表(例如流程图)列出思维过程,这会使你的思维显得很清晰,让人觉得一目了然。

(注意:问题分析这一部分放置的位置比较灵活,可以放在问题重述后面作为单独的一节(见到的频率最高),也可以放在模型假设和符号说明后面作为单独的一节,还可以针对每个问题将其写在模型建立中。)

%--------------------------模型假设--------------------------
\section{模型假设}
以下是6类常见的模型假设:
\begin{enumerate}
	\item 题目明确给出的假设条件
	\item 排除生活中的小概率事件(例如黑天鹅事件、非正常情况)
	\item 仅考虑问题中的核心因素,不考虑次要因素的影响
	\item 使用的模型中要求的假设
	\item 对模型中的参数形式(或者分布)进行假设
	\item 和题目联系很紧密的一些假设,主要是为了简化模型
\end{enumerate}

%--------------------------符号说明--------------------------
\section{符号说明}
%使用三线表格最好~
\begin{table}[h]
	\centering
	\begin{tabular}{p{2.0cm}<{\centering}p{9.0cm}<{\centering}p{2.0cm}<{\centering}}
		%指定单元格宽度, 并且水平居中。
		\hline
		符号 & 说明 & 单位 \\ %换行 
		\hline
		$\int$ & 积分符号 &  \\ %把你的符号写在这
		$W_0$ & 区分高峰和低峰的一个临界值 &  \\ %把你的符号写在这
		$M_t$ &  简单移动平均项 &  \\ %把你的符号写在这
		\hline
	\end{tabular}
\end{table}
本部分是对模型中使用的重要变量进行说明,一般排版时要放到一张表格中。

注意:不需要把所有变量都放到这个表里面,模型中用到的临时变量可以不放。下文中首次出现这些变量时也要进行解释,不然会降低文章的可读性。

%----------------------模型的建立与求解----------------------
\section{模型的建立与求解}

注意:这个部分里面的标题可根据论文内容进行调整

\subsection{问题一模型的建立与求解}
\subsubsection{模型的建立}
模型建立是将原问题抽象成用数学语言的表达式,它一定是在先前的问题分析和模型假设的基础上得来的。因为比赛时间很紧,大多时候我们都是使用别人已经建立好的模型。这部分一定要将题目问的问题和模型紧密结合起来,切忌随意套用模型。我们还可以对已有模型的某一方面进行改进或者优化,或者建立不同的模型解决同一个问题,这样就是论文的创新和亮点。
\subsubsection{模型的求解}
把实际问题归结为一定的数学模型后,就要利用数学模型求解所提出的实际问题了。一般需要借助计算机软件进行求解,例如常用的软件有Matlab, Spss, Lingo, Excel, Stata, Python等。求解完成后,得到的求解结果应该规范准确并且醒目,若求解结果过长,最好编入附录里。(注意:如果使用智能优化算法或者数值计算方法求解的话,需要简要阐明算法的计算步骤)
\subsection{问题二模型的建立与求解}

\subsection{问题三模型的建立与求解}

\begin{defn}\label{D1.1}
	设 $ x,y \in \mathbb{N} $,自然数的加法用符号“+”表示,且满足:
	\[ \left\{ \begin{array}{l}
		x+0 = x, \\
		x+y' = (x+y)'. \\
	\end{array} \right. \]
\end{defn}

\begin{figure}
	\centering
	\includegraphics[width=0.5\textwidth]{figure/no_homework}
	\caption{caption}
\end{figure}

\begin{thm}[定理名]
	$ 1+1 = 2 $.
\end{thm}

\begin{proof}
	由定义 \ref{D1.1},
	$ 1+1 = 1+0' = (1+0)' = 1' = 2 $.
\end{proof}

%----------------------模型的分析与检验----------------------
\section{模型的分析与检验}

模型的分析与检验的内容也可以放到模型的建立与求解部分,这里我们单独抽出来进行讲解,因为这部分往往是论文的加分项,很多优秀论文也会单独抽出一节来对这个内容进行讨论。

模型的分析 :在建模比赛中模型分析主要有两种,一个是灵敏度(性)分析,另一个是误差分析。灵敏度分析是研究与分析一个系统(或模型)的状态或输出变化对系统参数或周围条件变化的敏感程度的方法。其通用的步骤是:控制其他参数不变的情况下,改变模型中某个重要参数的值,然后观察模型的结果的变化情况。误差分析是指分析模型中的误差来源,或者估算模型中存在的误差,一般用于预测问题或者数值计算类问题。

模型的检验:模型检验可以分为两种,一种是使用模型之前应该进行的检验,例如层次分析法中一致性检验,灰色预测中的准指数规律的检验,这部分内容应该放在模型的建立部分;另一种是使用了模型后对模型的结果进行检验,数模中最常见的是稳定性检验,实际上这里的稳定性检验和前面的灵敏度分析非常类似。

%--------------------模型的评价、改进与推广--------------------
\section{模型的评价}
注:本部分的标题需要根据你的内容进行调整,例如:如果你没有写模型推广的话,就直接把标题写成模型的评价与改进。很多论文也把这部分的内容直接统称为“模型评价”部分,也是可以的。

\subsection{模型的优点}
优缺点是必须要写的内容,改进和推广是可选的,但还是建议大家写,实力比较强的建模者可以在这一块充分发挥,这部分对于整个论文的作用在于画龙点睛。
\subsection{模型的缺点}
缺点写的个数要比优点少
\subsection{模型的改进}
主要是针对模型中缺点有哪些可以改进的地方\cite{risken1996fokker};
\subsection{模型的推广}
将原题的要求进行扩展\cite{rossler1979equation},进一步讨论模型的实用性和可行性\cite{mckean1970nagumo}。

%----------------------------致谢----------------------------
%\begin{ack}
%	论文的致谢.
%\end{ack}

%--------------------------参考文献--------------------------
%\bibliography{reference}	%或者使用这个显示“参考文献”四个字
\bibliographystyle{elsarticle-num-names}
\begin{thebibliography}{60}
	\bibitem{1} 作者,\emph{题目},期刊,\textbf{卷号}(年份),页码.
\end{thebibliography}

%----------------------------附录----------------------------
%表格和小标题两种方案自选,个人喜欢用小标题
%小标题:
\newpage
\section*{附录}
\appendix				% 标题从 A 开始编号
\setcounter{section}{0}	% 如果你想从 1 开始编号,使用这个(不用也无需删除)
\setcounter{table}{0}   % 从 0 开始编号,显示出来表会 A1 开始编号
\setcounter{figure}{0}	% 定义编号格式,在数字序号前加字符“A"
\setcounter{equation}{0}% 公式也重新计数
\renewcommand{\thetable}{A\arabic{table}}
\renewcommand{\thefigure}{A\arabic{figure}}
\renewcommand\theequation{A.\arabic{equation}}

\section{MATLAB}
\subsection{function}
\lstinputlisting{input/code/cmb.m}

%表格:
%\begin{table}[htbp]
%	\centering
%	\begin{tabular}{|p{14.0cm}|}
%		%指定单元格宽度, 并且水平居中。
%		\hline
%		\textbf{附录1} \\ %换行 
%		\hline
%		介绍:支撑材料的文件列表  \\ 
%		\\
%		\\
%		\\
%		\hline
%	\end{tabular}
%\end{table}
%
%\begin{table}[htbp]
%	\centering
%	\begin{tabular}{|p{14.0cm}|}
%		%指定单元格宽度, 并且水平居中。
%		\hline
%		\textbf{附录2} \\ %换行 
%		\hline
%		介绍:该代码是某某语言编写的,作用是什么   \\ 
%		\\
%		\\
%		\\
%		\hline
%	\end{tabular}
%\end{table}

除了支撑材料的文件列表和源程序代码外,附录中还可以包括下面内容:
\begin{itemize}
	\item 某一问题的详细证明或求解过程;
	\item 自己在网上找到的数据;
	\item 比较大的流程图;
	\item 较繁杂的图表或计算结果
\end{itemize}
	
\end{document}
